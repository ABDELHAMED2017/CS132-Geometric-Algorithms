\documentclass[11pt]{article}
\pagestyle{empty}

\setlength{\textheight}{8.5in}
\setlength{\topmargin}{0.5in}
\setlength{\headheight}{0in}
\setlength{\headsep}{0in}
% %% \setlength{\footheight}{0in}
\setlength{\oddsidemargin}{0in}
\setlength{\textwidth}{6.5in}

\usepackage{times,url}

\begin{document}
\sloppy 
\begin{center}
\LARGE CAS CS 132\\
\Large Geometric Algorithms\\
\Large\rm Spring 2015\\~\\
\end{center}

\noindent{\large\bf Meeting Place:} CAS 313\\[\baselineskip]
\noindent{\large\bf Meeting
Time:} TR 3:30 -- 5:00 pm 
\\[\baselineskip] 

\noindent{\large\bf Instructor:} Prof.\ Mark Crovella\\[0.75\baselineskip]
\begin{minipage}[t]{0.60\textwidth}
\begin{itemize}
\item {\bf Office:} MCS-140E
\item {\bf Office Hours:} {\small W 3-4,  F 2-3}
\item {\bf Email:} crovella@bu.edu
\end{itemize}
\end{minipage}
~\\~\\~\\~\\
 \noindent{\large\bf Teaching Fellow:} Mr.\ Qinxun Bai\\[0.75\baselineskip]
 \begin{minipage}[t]{0.60\textwidth}
 \begin{itemize}
 \item {\bf Office Hours:} {\small T 5-6:30, F 4-5:30}
 \item {\bf Office Hours Location:} MCS B29 
 \item {\bf Lab Tutoring Hours:} {\small R 5-6, F 2-3}
 \item {\bf Email:} qinxun@bu.edu
 \end{itemize}
 \end{minipage}

\section*{Overview of the Course}

This course will introduce you to linear algebra from an algorithmic
standpoint.  Linear algebra is such a useful tool that it is crucially
important to a number of areas in Computer Science. For example, if you study
optimization, the starting point is linear algebra. If you study
computer graphics, the language you use every day is linear algebra. If
you study the performance of computer systems, you need linear
algebra. If you study algorithms -- especially graph algorithms -- you
will absolutely need linear algebra. If you study data mining, you will
use linear algebra all the time. 

The dominance of linear algebra arises because it is so fundamental, and
in some ways, very simple. It deals with objects that almost always can
be interpreted geometrically. So often we can use linear algebra in a
very intuitive manner -- so much so that many times it is actually the
best way to think about geometric problems. But it is also rigorous and so
captures situations that sometimes we would overlook if we were
proceeding purely intuitively. This is because it is also about solving
equations, and finding solutions to various kinds of problems. So the
advantage of being basic and fundamental is that it can be used and
applied in so many ways. 

\section*{Readings} 

The textbook for the course is David C. Lay, \emph{Linear Algebra and
  Its Applications} (LAA), 4th edition.    Many assignments will be taken
from the book.

\section*{Web Resources} 

The slides I use are actually executable python scripts, using the
\texttt{ipython notebook.}   If you have ipython notebook, you can
download and execute the examples on your own computer, and you can
modify them any way you'd like, play around with them, experiment, etc.

The slides I use in lecture are published on \texttt{github.}   The
repository is
\url{https://github.com/mcrovella/CS132-Geometric-Algorithms}.  Mr.\ Bai
will describe how to access the repository using \texttt{git,} but you
can simply download directly from the web site if you prefer.
 

\section*{Reading and Homeworks}

\begin{enumerate}
\item  You have about 10-12 pages of reading for each class.   Class will be
more understandable if you do the reading first.   A good plan is to set aside
time on Mondays and Wednesdays to do readings.
\item  Homeworks will be assigned on Thursdays.
\item Homeworks are due at 3:30 pm on Tuesdays.  This means they are due
before the start of Tuesday's class.
\item You can discuss homeworks in section meeting on Mondays.   But don't
expect that Mr.\ Bai will be going into detail -- instead, he will
answer specific questions!
\item Homeworks will be submitted via \emph{websubmit} (instructions in class).
\end{enumerate}

\section*{Piazza}

We will be using Piazza for class discussion. The system is highly
tuned to getting you help fast and efficiently from classmates, Mr.\ Bai,
and myself. Rather than emailing questions to the teaching staff,
I encourage you to post your questions on Piazza.   Our class Piazza
page  is at: \url{https://piazza.com/bu/spring2015/cs132/home}. 
We will also use Piazza for distributing materials
such as homeworks and solutions.

When someone posts a question on Piazza, if you know the answer, please
go ahead and post it.   However pleased \emph{don't} provide answers to homework
questions on Piazza.   It's OK to tell people \emph{where to look} to
get answers, or to correct mistakes;  just don't provide actual solutions
to homeworks.

\section*{Programming Environment}

We will use \texttt{python} as the language for teaching and for
assignments that require coding.  

\section*{Clickers}

We will be using ``Peer instruction'' as part of the lectures.  This
requires you to answer occasional questions during lecture, sometimes
after discussion with your classmates.   

To support this, we will use clickers for student feedback during lecture.  If you
don't already have one, you need to get one.  It is the Turning
Technologies Response Card RF (ISBN 9781934931394).  This is the
standard ``BU clicker'' and you can get it at
the BU Bookstore.   You will need to bring it to every lecture.
% register it on Blackboard 

Also, to encourage this interactive style of lecture, I will ask you to
put away laptops and phones during lecture.    Please note that I will
post copies of lecture slides online for you to consult after class if
you like.

You will also want to bring pencil and paper to lecture.   This isn't
absolutely critical, but you will find it easier if you can jot a note
or two while responding to clicker questions.

\section*{Course and Grading Administration}

Assignments will be submitted using \texttt{websubmit}.   Mr.\ Bai will
explain how to submit assignments.  

\emph{NOTE: IMPORTANT:} Late assignments \textbf{WILL NOT} be accepted.   However, your final
grade will be based on the top 10 homeworks submitted (out of 12).   

Final grades will be computed based on the following:
\begin{description}
\item[60\%] Homework assignments.  The top 10 homework grades (out of the
  12 assigned) will be used to compute this score.
\item[5\%] Attendance and In-class participation via clicker.
\item[10\%] Midterm
\item[25\%] Final (Cumulative)
\end{description}

To get full credit for class participation by clicker, you need to use
the clicker on 85\% of the questions that are posed in lecture.   So if
you miss a question here or there, or forget your clicker one day, don't
worry as long as you come to lecture consistently.

The exact cutoffs for final grades will be determined after the class is
complete, but you can assume that they will approximately follow the
typical pattern in which scores of 85-100 are in the A range (A-, A,
A+), 70-85 are in the B range (B-, B, B+), and 60-70 are in the
``counting'' C range (C, C+).

This grading policy implies the following:
\begin{enumerate}
\item You need to consistently work the problem sets each week.   Plan
  to set aside a regular time each week to do them.
\item The final, which is cumulative, is important to getting a good
  final grade.   You could conceivably miss or fail the midterm and
  still do reasonably well (but I don't recommend counting on this strategy).
\end{enumerate}

\newpage

\section*{Academic Honesty}

You may discuss homework assignments with classmates, but you are 
solely responsible for what you turn in. Collaboration in the form of
discussion is allowed, but all forms of cheating (copying parts of a
classmate's assignment, plagiarism from books or old posted solutions)
are NOT allowed. We -- both teaching staff and students -- are expected
to abide by the guidelines and rules of the Academic Code of Conduct
(which is at
\url{http://www.bu.edu/dos/policies/student-responsibilities/}).

You can probably, if you try hard enough, find solutions for homework
problems online.    Given the nature of the Internet, this is
inevitable.   Let me make a couple of comments about that:
\begin{enumerate}
\item If you are looking online for an answer because you don't know how
  to start thinking about a problem, talk to Mr.\ Bai or myself, who may be
  able to give you pointers to get you started.  Piazza is great for
  this -- you can usually get an answer in an hour if not a few minutes.
\item If you are looking online for an answer because you want to see if
  your solution is correct, ask yourself if there is some way to verify
  the solution yourself.   Usually, there is.  You will understand what you have done
  \emph{much} better if you do that.
\item If you are looking online for an answer because you don't have
  enough time and are getting close to the assignment deadline, think about this:
  \begin{enumerate}
  \item what you are doing is intellectually dishonest,
  \item you are going to have to solve problems like this on the midterm
    and final, and
  \item you can miss up to two homeworks without penalty.
  \end{enumerate}
So ... it would be better to simply submit what you have at the deadline
(without going online to cheat) and plan to allocate more time for
homeworks in the future.
\end{enumerate}

\newpage
\section*{Course Schedule}

\textbf{IMPORTANT:} It is important to do the reading \emph{before} the class on
which it is based.   All readings are from LAA.
\\~\\
\small
\begin{centering}
\begin{tabular}{||l|p{3in}|l|l|l||}
\hline\hline
Date & Topics  & Reading & Assigned & Due  \\
\hline\hline
% Lay 1.1
1/20 & NO CLASS &&& \\
1/22 & 1: Linear Equations & 1.1 & H1  & \\
\hline

% Lay 1.2, 1.3
1/27 & 2: Row Reduction & 1.2 & & H1 \\
1/29 & 3: Vector Equations & 1.3 & H2 & \\
\hline

%  Lay 1.4, 1.5
2/3 & 4: $Ax = b$ & 1.4 & & H2 \\
2/5 & 5: Solution Sets & 1.5, 1.6 & H3 & \\
\hline

%  Lay 1.7, 1.8
2/10 & 6: Linear Independence & 1.7 & & H3 \\
2/12 & 7: Linear Transformations & 1.8 & H4 & \\
\hline

% Lay 2.1
2/17 & Substitute Monday - NO CLASS & & &\\
2/19 & 8: Matrix Operations  & 2.1 & H5 & H4 \\
\hline

%  Lay 2.2, 2.3
2/24 & 9: Matrix Inverse & 2.2 & & H5\\
2/26 & 10: Invertible Matrices & 2.3 & H6 &\\
\hline

% Lay 4.1, 4.2
3/3 & 11: Vector Spaces & 4.1 & & H6\\
3/5 & 12: Null and Column Spaces & 4.2 & &\\
\hline

3/10 & Spring Break &&&\\
3/12 & Spring Break &&&\\
\hline

% Lay 4.3
3/17 & Midterm & & &\\
3/19 & 13: Bases & 4.3 & H7 &\\
\hline

% Lay 4.4, 4.5
3/24 & 14: Coordinate Systems & 4.4 & & H7\\
3/26 & 15: Dimension & 4.5 & H8 &\\
\hline

% Lay 4.6, 5.1
3/31 & 16: Rank & 4.6 &  & H8\\
4/2 & 17: Eigenvectors and Eigenvalues & 5.1 & H9 &\\
\hline

% Lay 5.2, 5.3
4/7 & 18: Characteristic Equation & 5.2 & & H9\\
4/9 & 19: Diagonalization & 5.3 & H10 &\\
\hline

% Lay 6.1, 6.2
4/14 & 20: Inner Product & 6.1 & & H10\\
4/16 & 21: Orthogonal Sets & 6.2 & H11 &\\
 \hline

% Lay 6.3, 6.5
4/21 & 22: (Guest lecture) Projections & 6.3 & & H11\\
4/23 & 23: Least Squares & 6.5 & H12 & \\
\hline

% Lay 7.1, 7.2
4/28 & 24: Symmetric Matrices & 7.1 & & H12\\
4/30 & 25: Quadratic Forms & 7.2 &  &\\
\hline\hline

\end{tabular}\\
\end{centering}


\end{document}
