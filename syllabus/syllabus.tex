\documentclass[11pt]{article}
\pagestyle{empty}

\setlength{\textheight}{8.5in}
\setlength{\topmargin}{0.5in}
\setlength{\headheight}{0in}
\setlength{\headsep}{0in}
% %% \setlength{\footheight}{0in}
\setlength{\oddsidemargin}{0in}
\setlength{\textwidth}{6.5in}

\usepackage{times,url}

\begin{document}
\begin{center}
\LARGE CAS CS 132\\
\Large Object-Oriented Programming\\
\Large\rm Spring 2015\\~\\
\end{center}

\noindent{\large\bf Meeting Place:} ??\\[\baselineskip]
\noindent{\large\bf Meeting
Time:} TR 3:30 -- 5:00 pm 
\\[\baselineskip] 

\noindent{\large\bf Instructor:} Professor Mark Crovella\\[0.75\baselineskip]
\begin{minipage}[t]{0.60\textwidth}
\begin{itemize}
\item {\bf Office:} MCS-140E
\item {\bf Office Hours:} {\small TBD}
\item {\bf Email:} crovella@bu.edu
\end{itemize}
\end{minipage}
~\\~\\~\\~\\
 \noindent{\large\bf Teaching Fellow:} QInxun Bai\\[0.75\baselineskip]
 \begin{minipage}[t]{0.60\textwidth}
 \begin{itemize}
 \item {\bf Office:} TBD
 \item {\bf Office Hours:} {\small TBD}
 \item {\bf Email:} qinxun@bu.edu
 \end{itemize}
 \end{minipage}

\section*{Overview of the Course}

This course will ...

\section*{Readings} 

The textbook for the course is David C. Lay, \emph{Linear Algebra and
  Its Applications} (LAA), 4th edition.    Assignments will be taken
from the book.

\section*{Web Resources} 

The slides I use are actually executable python scripts, using the
\texttt{ipython notebook.}   If you have ipython notebook, you can
download and execute the examples on your own computer, and you can
modify them any way you'd like, play around with them, experiment, etc.

The slides I use in lecture are published on \texttt{github.}   The
repository is
\texttt{https://github.com/mcrovella/CS132-Geometric-Algorithms}.  The TF
will describe how to access the repository using \texttt{git,} but you
can simply download directly from the web site if you prefer.
 
We will be using Piazza for class discussion. The system is highly
tuned to getting you help fast and efficiently from classmates, the
TF, and myself. Rather than emailing questions to the teaching staff,
I encourage you to post your questions on Piazza.   Our class Piazza
page  is at: \url{https://piazza.com/bu/...}. 

\section*{Grades}

\section*{Programming Environment}

We will use \texttt{python} as the language for teaching and for
submitting assignments that require coding.



\section*{Course and Grading Administration}


Assignments will be submitted using \texttt{websubmit}.   Assignments will
generally be due \emph{WHEN?}

\emph{LATE POLICY: You have a total of three late days that you can use without penalty.
After you have used your three late days, each day reduces the
assignment grade by one step (eg, from check-plus to check, etc).}

\sloppypar
Lecture slides, homework assignments, and this syllabus will be available
online on the website.  Incompletes will not be given. 

\section*{Assignments}

There will be weekly assignments.

\section*{Academic Honesty}
One of the goals of this course is to provide you with an intensive
programming experience that will raise your level of programming
skills.  You will come out of this course with the ability to take on
larger programming projects than you could before.  

Hence this is a programming-intensive course;  almost all of your grade will
be based on code that you submit.   

Some of the homework assignments given in this course were originally developed
at other institutions.  Undoubtedly, you will be able to find examples of
assignment solutions online.   Likewise, your classmates will be solving
the same assignments as you.

I have two messages with respect to academic honesty in this course: (1)
submitting someone else's code means you lose about 90\% of the value of
being in the course at all;  and (2) you will probably get caught, which
will have very serious consequences.

This doesn't mean you shouldn't ask for help;  what it means is that
\emph{you must indicate on your submission any help you received.}  That
includes discussions with the TF, grader, or other students.  Do this in
the comments at the beginning of the code.

This discussion should make clear that \emph{you must not share code
  with other students.}  Don't ask for someone's code, and don't provide
  it.  Discuss ideas and strategies freely, but write your own code.

Also, \emph{you must not look a solutions from other courses or other
  years.}  The assignments in this course will be different in some ways
  from other courses and years, so using ``found'' code in this way is
  dangerous as well as being dishonest.

To back this up, keep in mind two things:  first, you must be prepared
to explain any program code you submit.   The TF, the grader, and I may
ask any of you to explain your code at any time.   And finally, I use
automated plagiarism detection tools.  These tools compare code between
students, as well as code that is available online.  I have used these
tools for some time and (unfortunately) they regularly turn up cases of
academic dishonesty.

\newpage
\section*{Syllabus}

\small
\begin{centering}
\begin{tabular}{||l|p{3in}|l|l||}
\hline\hline
Date & Topics  & Assigned & Due  \\
\hline\hline
% Lay 1.1
1/20 & NO CLASS && \\
1/22 & 1: Linear Equations &  & \\
\hline

% Lay 1.2, 1.3
1/27 & & & \\
1/29 & & & \\
\hline

%  Lay 1.4, 1.5
2/3 & & & \\
2/5 & & & \\
\hline

%  Lay 1.7, 1.8
2/10 & & & \\
2/12 & & & \\
\hline

% Lay 2.1
2/17 & Substitute Monday - No Classes & & \\
2/19 & & & \\
\hline

%  Lay 2.2, 2.3
2/24 & & & \\
2/26 & & & \\
\hline

% Lay 4.1, 4.2
3/3 & & & \\
3/5 & & & \\
\hline

3/10 & Spring Break &&\\
3/12 & Spring Break &&\\
\hline

% Lay 4.3
3/17 & Midterm & & \\
3/19 & & & \\
\hline

% Lay 4.4, 4.5
3/24 & & & \\
3/26 & & & \\
\hline

% Lay 4.6, 5.1
3/31 & & & \\
4/2 & & & \\
\hline

% Lay 5.2, 5.3
4/7 & & & \\
4/9 & & & \\
\hline

% Lay 6.1, 6.2
4/14 & & & \\
4/16 & & & \\
 \hline

% Lay 6.4, 6.5
4/21 & (Guest lecture) & & \\
4/23 & & & \\
\hline

% Lay 7.1, 7.2
4/28 & & & \\
4/30 & & & \\
\hline\hline

\end{tabular}\\
\end{centering}


\end{document}
